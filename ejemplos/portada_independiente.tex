\documentclass[10pt]{book}

\usepackage[utf8]{inputenc}        % Acentos directo en LaTeX.
\usepackage[T1]{fontenc}           % Hyphenation para palabras con acentos, y uso de \textquotedbl.
\usepackage[spanish,mexico]{babel} % Paquete para utilizar el idioma español, con el texto "tabla".
\usepackage{mdframed}              % Para colocar una caja de color (usada en portada).
\usepackage{changepage}            % Para ajustar el margen del código en portada.
\usepackage[svgnames]{xcolor}      % Colores para el código, portada, y otros.
\usepackage{geometry}              % Permite modificar los márgenes del documento.
\usepackage{listings}              % Para incluir código fuente.
\usepackage{graphicx}              % Paquete para incluir imágenes y usar scalebox.

% Declaración de colores de la pantalla.
\colorlet{FondoPortada}{DeepPink!40!Black}
\colorlet{TextoPortada}{White}
\colorlet{FondoCodigoPortada}{Black!85!DeepPink}

% Márgenes usados tanto en el paquete geometry como para la compensación del listado en portada.
\newcommand{\margenIzquierdo}{0.6875in} % Margen a la izquierda del documento.
\newcommand{\margenDerecho}{0.6875in} % Margen a la derecha del documento.

% Definir márgenes del documento.
\geometry{
	letterpaper,
    tmargin=0.625in, % Margen superior.
    bmargin=0.80in, % Margen inferior.
    lmargin=\margenIzquierdo, % Margen izquierdo.
    rmargin=\margenDerecho % Margen derecho.
}

% Soporte de caracteres en español para los listados:
\lstset{
    literate= % Lista de símbolos en español que marcarían un error.
    {á}{{\'a}}1  {é}{{\'e}}1  {í}{{\'i}}1  {ó}{{\'o}}1  {ú}{{\'u}}1
    {Á}{{\'A}}1  {É}{{\'E}}1  {Í}{{\'I}}1  {Ó}{{\'O}}1  {Ú}{{\'U}}1
    {à}{{\`a}}1  {è}{{\`e}}1  {ì}{{\`i}}1  {ò}{{\`o}}1  {ù}{{\`u}}1
    {À}{{\`A}}1  {È}{{\'E}}1  {Ì}{{\`I}}1  {Ò}{{\`O}}1  {Ù}{{\`U}}1
    {ä}{{\"a}}1  {ë}{{\"e}}1  {ï}{{\"i}}1  {ö}{{\"o}}1  {ü}{{\"u}}1
    {Ä}{{\"A}}1  {Ë}{{\"E}}1  {Ï}{{\"I}}1  {Ö}{{\"O}}1  {Ü}{{\"U}}1
    {â}{{\^a}}1  {ê}{{\^e}}1  {î}{{\^i}}1  {ô}{{\^o}}1  {û}{{\^u}}1
    {Â}{{\^A}}1  {Ê}{{\^E}}1  {Î}{{\^I}}1  {Ô}{{\^O}}1  {Û}{{\^U}}1
    {œ}{{\oe}}1  {Œ}{{\OE}}1  {æ}{{\ae}}1  {Æ}{{\AE}}1  {ß}{{\ss}}1
    {ű}{{\H{u}}}1{Ű}{{\H{U}}}1{ő}{{\H{o}}}1{Ő}{{\H{O}}}1
    {ç}{{\c c}}1 {Ç}{{\c C}}1 {ø}{{\o}}1   {å}{{\r a}}1 {Å}{{\r A}}1
    {£}{{\pounds}}1           {«}{{\guillemotleft}}1
    {»}{{\guillemotright}}1   {ñ}{{\~n}}1  {Ñ}{{\~N}}1 {¿}{{?`}}1
}
% Estilo del código LaTeX utilizado en la portada.
\lstdefinestyle{portada}{
    language=[LaTeX]{TeX},
    numbers=none,
    frame={},
    basicstyle=\LARGE\ttfamily\color{TextoPortada},
    backgroundcolor={},
    keywordstyle=\color{DeepPink!50},
    identifierstyle=\color{TextoPortada},
    lineskip=20pt,
}

\begin{document}

\begin{titlepage} % Elimina encabezado y pie de página.
\pagecolor{FondoPortada}
\color{TextoPortada}

% Espacio en blanco al inicio de la página.
% Se requiere usar \vspace* porque la instrucción \vspace no pone el espacio
% si es lo primero que hay en la página (como en este caso).
\vspace*{2.5cm}

%------------------------------------------------
%   Título de nuestra magna obra.
%------------------------------------------------
\noindent\scalebox{9}{\textbf{\LaTeX{}}}
\newline\vspace{0.5cm}
\noindent\scalebox{4.5}{para tesis de ingeniería}

% Espacio entre el título y el código.
\vspace{2.5cm}

% Usamos el entorno adjustwidth para que el código ocupe todo el ancho de la hoja.
% Es decir, le restamos el margen izquierdo y el margen derecho.
% https://tex.stackexchange.com/a/594
\begin{adjustwidth}{-\margenIzquierdo}{-\margenDerecho}

% El entorno mdframed nos sirve para colocar un recuadro de color debajo del código.
% Debido a que el backgroundcolor de un listado se puede comportar extraño con el
% interlineado, el recuadro con mdframed nos da un resultado uniforme.
\begin{mdframed}[
	backgroundcolor=FondoCodigoPortada,
	innertopmargin=0pt,
	innerbottommargin=0pt,
	innerrightmargin=0pt,
	innerleftmargin=0pt
]
% Código a ser mostrado en portada, dentro del recuadro de color.
\begin{lstlisting}[style=portada]
   \usepackage[spanish,mexico]{babel}
   \begin{portada}
      \author{Carlos Ramos}
      \title{\LaTeX{} para tesis de ingeniería}
   \end{portada}
\end{lstlisting}
% Fin del recuadro de color.
\end{mdframed}
% Fin de la eliminación de los márgenes.
\end{adjustwidth}

% Espacio del código a la línea horizontal.
\vspace{2.5cm}

% Regla horizontal del 25% del ancho del texto, de un alto de 2pt.
\noindent\rule{0.25\linewidth}{2pt}

% Espacio entre la regla y el texto que va debajo.
\vspace{1.25cm}

% Textito al final de la página.
\noindent\textbf{\Large
	Imágenes, ecuaciones, código, y referencias
	\\[4pt] % Espacio entre cada línea.
	Redacta tu tesis sin dolor de cabeza
}

% Fin de la página de título.
\end{titlepage}

% Regresar el color de la página.
\pagecolor{White}

% Aquí podría empezar el contenido del documento.

\end{document}