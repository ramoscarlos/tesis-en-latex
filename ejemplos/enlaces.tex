\documentclass{article}
\usepackage[utf8]{inputenc}
\usepackage[spanish,mexico]{babel} % Paquete para utilizar el idioma español.
\usepackage{hyperref}              % Paquete para usar vínculos.

\title{Enlaces}
\author{Carlos Ramos}
\date{Enero 2021}

\begin{document}

\maketitle

\section{Esta es una sección de ejemplo}
\label{sec:ejemplo}

La sección anterior solo está para demostrar que si la referenciamos mediante un \texttt{\textbackslash{}ref\{sec:ejemplo\}} no se ve del todo bien: \ref{sec:ejemplo}.

\subsection{Una subsección}
\label{sub:ejemplo}

La subsección \ref{sub:ejemplo} solo sirve para seguir demostrando el punto: un cuadro rojo que no se ve bonito, tal es el comportamiento predeterminado de \texttt{hyperref}.

\section{Tablas}
\label{sec:tablas}

Aquí está la tabla \ref{tab:ejemplo}, misma que no

\begin{table}[ht!]
	\centering
	\begin{tabular}{lll}
		x & o & x \\
		o & x & o \\
		o & o & x
	\end{tabular}
	\caption{Ejemplo de tabla a referenciar.}
	\label{tab:ejemplo}
\end{table}

\end{document}