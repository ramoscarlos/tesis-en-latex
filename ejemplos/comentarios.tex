\documentclass{article}
\usepackage[utf8]{inputenc}
\usepackage{comment}        % Sirve para usar el entorno "comment"

\title{Ejemplo de comentarios}
\author{Carlos Ramos}
\date{Noviembre 2020}

\begin{document}
\maketitle

% Aquí va un comentario desde el inicio de la línea.

Este texto sí se imprime %, pero esto ya no porque es comentario.

Incluso con % otro porcentaje % ya no podemos regresar a texto normal.

\begin{comment}
Y todo esto es un comentario, que puede que se vea como comentario,
o no, dependiendo del editor que estés utilizando.

Podemos meter muchas instrucciones de \LaTeX{} aquí, pero igual no se
van a tomar en cuenta.

\textbf{Esto debería estar en negritas, pero está comentado}
\end{comment}

Es difícil ver el resultado en PDF, pero confío en que verás
el código fuente.

\end{document}