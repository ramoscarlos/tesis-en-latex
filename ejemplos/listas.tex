\documentclass{article}
\usepackage[utf8]{inputenc}
\usepackage[spanish,mexico]{babel} % Paquete para utilizar el idioma español.

\title{Listas}
\author{Carlos Ramos}
\date{Noviembre 2020}

\begin{document}

\maketitle

\section{Listas sin orden}

Esto es una lista sin orden:

\begin{itemize}
	\item Primer elemento.
	\item Segundo elemento.
	\item No importa el número.
	\item Son viñetas y ya.
\end{itemize}

\section{Listas con orden}

\begin{enumerate}
	\item Primer elemento.
	\item Segundo elemento.
	\item Aquí sí hay números.
	\item Cuatro, cuatro.
\end{enumerate}

\section{Listas anidadas}

\begin{enumerate}
	\item Primera lista.
	\item Sigue siendo la misma.
	\item Aquí abro otra:
	\begin{itemize}
		\item Segundo nivel.
		\item Todavía segundo...
		\item Sigo bajando:
		\begin{itemize}
			\item Tercer nivel.
			\item 2do sin orden.
			\item Bajamos:
			\begin{enumerate}
				\item Ordenada.
				\item Otro dos.
			\end{enumerate}
		\end{itemize}
		\item Vuelvo a 2do nivel.
	\end{itemize}
	\item Elemento cuatro.
\end{enumerate}

\end{document}