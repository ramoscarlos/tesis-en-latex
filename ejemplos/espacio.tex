\documentclass{article}
\usepackage[utf8]{inputenc}
\usepackage[spanish,mexico]{babel} % Paquete para utilizar el idioma español.

\title{Espacio en blanco}
\author{Carlos Ramos}
\date{Noviembre 2020}

\begin{document}

\maketitle

\section{Demostración del espacio en blanco}

En este primer fragmento tenemos mucho espacio en blanco en el código fuente, mismo que no se visualiza en el documento generado:

[Hola,            aquí         hay         mucho,          	
pero mucho]       			
        	
           		
[           espacio                    ]
       	
            	
                    	
[				¿No crees?]   	

~\newline
\indent \^{} Agregamos una nueva línea aquí para que nos permita separar visualmente lo anterior de este nuevo párrafo. Requiere la tilde inicial para que no se queje de que ``no hay nada que terminar'' o algo así. Le damos un espacio en blanco para que la línea tenga contenido y pueda darnos una nueva línea.

~\newline
Sigamos. Por otro lado, en este nuevo ejemplo se utiliza el símbolo de la tilde en lugar de un espacio para que \LaTeX{} nos haga el favor de dejar los espacios tal y como los escribimos:

[Hola,~~~~~~~~~~~~aquí~~~~~~~~~hay~~~~~~~~~mucho~~~~~~~~~~~
pero mucho]~~~~~~~~~~~~~~~~~
\newline\newline
[~~~~~~~~~~~espacio~~~~~~~~~~~~~~~~~~~~]
\newline
\newline
[~~~~~~~~~~~~~~~~¿No crees?]


~\newline
Ahora la pregunta es, ¿cuándo inicia un nuevo párrafo y cuándo no?


% Lo volvemos a escribir, para demostrar que LaTeX hace párrafos naturales con dos Enter.
Ahora la pregunta es, ¿cuándo inicia un nuevo párrafo y cuándo no?

% Pero al forzar el salto de línea, no.
~\newline
\indent Ahora la pregunta es, ¿cuándo inicia un nuevo párrafo y cuándo no?

\end{document}