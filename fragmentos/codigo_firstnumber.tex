\begin{lstlisting}[
	style=latex,
	label=lst:listado_firstnumber,
	caption={Listado con \texttt{firstnumber} y \texttt{stepnumber} definidos.},
	firstnumber=414,
	stepnumber=4
]
linebackgroundcolor={
	\ifnum \value{lstnumber} =  5 \color{codigo_linea_resaltada}
	\else \ifnum \value{lstnumber} =  6 \color{codigo_linea_resaltada}
	\else \ifnum \value{lstnumber} =  7 \color{codigo_linea_resaltada}
	\else \ifnum \value{lstnumber} =  8 \color{codigo_linea_resaltada}
	\else \ifnum \value{lstnumber} =  9 \color{codigo_linea_resaltada}
	\else \color{codigo_fondo}
	\fi\fi\fi\fi\fi % Tantos \fi como líneas subrayadas.
}
\end{lstlisting}