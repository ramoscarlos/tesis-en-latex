% === División de palabras ====================================================
\hyphenation{el-e-ct-roenc-ef-al-og-raf-ista}

%%%%%%%%%%%%%%%%%%%%%%%%%%%%%%%%%%%%%%%%%%%%%%%%%%%%%%%%%%%%%%%%%%%%%%%%%%%%%%%
%%%%% Definición de constantes %%%%%%%%%%%%%%%%%%%%%%%%%%%%%%%%%%%%%%%%%%%%%%%%
%%%%%%%%%%%%%%%%%%%%%%%%%%%%%%%%%%%%%%%%%%%%%%%%%%%%%%%%%%%%%%%%%%%%%%%%%%%%%%%
% Las constantes aquí definidas se usan en más de una instrucción y, para evitar
% cambiar todas las instancias, mejor se declaran una vez para afectar todos los
% lugares donde se tengan que utilizar.

\newcommand{\espacioleyenda}{2pt} % Espacio entre leyenda y figuras, listados.

%%%%%%%%%%%%%%%%%%%%%%%%%%%%%%%%%%%%%%%%%%%%%%%%%%%%%%%%%%%%%%%%%%%%%%%%%%%%%%%
%%%%% Definición de colores usados en el documento (requiere xcolor) %%%%%%%%%%
%%%%%%%%%%%%%%%%%%%%%%%%%%%%%%%%%%%%%%%%%%%%%%%%%%%%%%%%%%%%%%%%%%%%%%%%%%%%%%%

% Definiendo un color en base a otros requiere \colorlet, no \definecolor.
%     https://tex.stackexchange.com/a/34913
%     https://texblog.org/2015/12/08/custom-colors-in-latex/
% Colores de prueba (amarrillo, verde limón, claro que sí).
\colorlet{ColorBase}{DeepPink}
\colorlet{Prueba1}{Gold}
\colorlet{Prueba2}{LimeGreen}
% Colores de la portada.
\colorlet{FondoPortada}{ColorBase!40!Black}
\colorlet{TextoPortada}{White}
\colorlet{FondoCodigoPortada}{ColorBase!15!Black}
% Colores de encabezado y pie de página.
\colorlet{PiePaginaNumeroPagina}{Black!85}
\colorlet{PiePaginaCapitulo}{Black!70}
% Colores del título de cada capítulo.
\colorlet{FondoCapitulo}{FondoPortada}
\colorlet{TextoCapitulo}{TextoPortada}
% Color de los enlaces.
\colorlet{Enlaces}{FondoPortada}
% Color del resaltado para instrucciones de LaTeX.
\definecolor{Resaltado}{rgb}{0.96, 0.96, 0.96}
% Colores para el código.
\colorlet{codigo_linea_margen}{ColorBase!50}
\colorlet{codigo_fondo}{White} % Cambiado a sin fondo para poder colocar \vdots sin blanco entre líneas.
\colorlet{codigo_numero_linea}{Black!45}
\colorlet{codigo_base}{Black!90}
\colorlet{codigo_cadena}{DarkGoldenrod!80}
\colorlet{codigo_comentarios}{Green!90!Black}
\colorlet{codigo_palabraclave}{Navy!70!Black} % Valor de impresión
% \colorlet{codigo_palabraclave}{Navy!80} % Valor de prueba para daltónicos.
\colorlet{codigo_identificador}{DarkRed!80}
\colorlet{codigo_linea_resaltada}{DarkGoldenrod!10}
% Color de las leyendas.
\colorlet{Leyendas}{Black!85}

%%%%%%%%%%%%%%%%%%%%%%%%%%%%%%%%%%%%%%%%%%%%%%%%%%%%%%%%%%%%%%%%%%%%%%%%%%%%%%%
%%%%% Configuración del paquete listings %%%%%%%%%%%%%%%%%%%%%%%%%%%%%%%%%%%%%%
%%%%%%%%%%%%%%%%%%%%%%%%%%%%%%%%%%%%%%%%%%%%%%%%%%%%%%%%%%%%%%%%%%%%%%%%%%%%%%%

% Cambio de los nombres utilizados a español.
\renewcommand{\lstlistingname}{Listado}
\renewcommand{\lstlistlistingname}{Índice de listados}
% Ajuste para aceptar caracteres del idioma español.
\lstset{
    literate= % Lista de símbolos en español que marcarían un error.
    {á}{{\'a}}1  {é}{{\'e}}1  {í}{{\'i}}1  {ó}{{\'o}}1  {ú}{{\'u}}1
    {Á}{{\'A}}1  {É}{{\'E}}1  {Í}{{\'I}}1  {Ó}{{\'O}}1  {Ú}{{\'U}}1
    {à}{{\`a}}1  {è}{{\`e}}1  {ì}{{\`i}}1  {ò}{{\`o}}1  {ù}{{\`u}}1
    {À}{{\`A}}1  {È}{{\'E}}1  {Ì}{{\`I}}1  {Ò}{{\`O}}1  {Ù}{{\`U}}1
    {ä}{{\"a}}1  {ë}{{\"e}}1  {ï}{{\"i}}1  {ö}{{\"o}}1  {ü}{{\"u}}1
    {Ä}{{\"A}}1  {Ë}{{\"E}}1  {Ï}{{\"I}}1  {Ö}{{\"O}}1  {Ü}{{\"U}}1
    {â}{{\^a}}1  {ê}{{\^e}}1  {î}{{\^i}}1  {ô}{{\^o}}1  {û}{{\^u}}1
    {Â}{{\^A}}1  {Ê}{{\^E}}1  {Î}{{\^I}}1  {Ô}{{\^O}}1  {Û}{{\^U}}1
    {œ}{{\oe}}1  {Œ}{{\OE}}1  {æ}{{\ae}}1  {Æ}{{\AE}}1  {ß}{{\ss}}1
    {ű}{{\H{u}}}1{Ű}{{\H{U}}}1{ő}{{\H{o}}}1{Ő}{{\H{O}}}1{¡}{{!`}}1
    {ç}{{\c c}}1 {Ç}{{\c C}}1 {ø}{{\o}}1   {å}{{\r a}}1 {Å}{{\r A}}1
    {€}{{\euro}}1{£}{{\pounds}}1           {«}{{\guillemotleft}}1
    {»}{{\guillemotright}}1   {ñ}{{\~n}}1  {Ñ}{{\~N}}1 {¿}{{?`}}1
}
% Opciones adicionales que afectarán a todos los listados del documento.
\lstset{
    language={},
    numbers=none,
    numbersep=5pt,
    numberstyle=\tiny\color{codigo_numero_linea},
    backgroundcolor=\color{codigo_fondo},
    basicstyle=\footnotesize\ttfamily\color{codigo_base},
    commentstyle=\itshape\color{codigo_comentarios},
    stringstyle=\color{codigo_cadena},
    keywordstyle=\bfseries\color{codigo_palabraclave},
    identifierstyle=\color{codigo_identificador},
    tabsize=4,
    captionpos=b,
    breaklines=true,
    frame=lines,
    rulecolor=\color{codigo_linea_margen},
    postbreak=\mbox{\textcolor{red}{$\hookrightarrow$}\space},
    upquote=true, % Tomado de https://tex.stackexchange.com/questions/166790/how-can-i-get-straight-double-quotes-in-listings
    abovecaptionskip=\espacioleyenda
    % Si se usa el escapechar | las tablas marcan error por el uso en alineación.
    % escapechar=|, %Tomado de https://tex.stackexchange.com/a/144171
}
% Estilo del código LaTeX utilizado exclusivamente en la portada.
\iflulu
\lstdefinestyle{portada}{
    language=[LaTeX]{TeX},
    frame={},
    basicstyle=\Large\ttfamily\color{TextoPortada},
    backgroundcolor={},
    keywordstyle=\color{ColorBase!50},
    identifierstyle=\color{TextoPortada},
    lineskip=20pt
}
\else
\lstdefinestyle{portada}{
    language=[LaTeX]{TeX},
    frame={},
    basicstyle=\LARGE\ttfamily\color{TextoPortada},
    backgroundcolor={},
    keywordstyle=\color{ColorBase!50},
    identifierstyle=\color{TextoPortada},
    lineskip=20pt
}
\fi
% Estilo del código LaTeX usado a lo largo del documento.
\lstdefinestyle{latex}{
    mathescape=true, % Usado para meter $\vdots$
    language=[LaTeX]{TeX},
    identifierstyle=\color{codigo_base},
    texcsstyle=*\bfseries\color{codigo_palabraclave},
    moretexcs={
    % Deberían estar pero no están.
    maketitle, part, subsection, subsubsection, paragraph, subparagraph,
    setlength, tableofcontents, addto, contentsname, listoffigures,
    listoftables, listfigurename, listtablename, guillemotright, guillemotright,
    % fontenc
    textquotedbl,
    % textcomp
    textquotesingle,
    % graphicx
    scalebox,graphicspath,includegraphics,
    % xcolor
    textcolor, colorlet, pagecolor, color, definecolor,
    % geometry
    geometry,
    % fancyhdr
    chaptermark, thechapter, chaptername,thepage,chapter,
    fancyhead, fancyfoot, fancypagestyle, headrule,
    % setspace
    doublespacing, onehalfspacing, setstretch,
    % caption.
    DeclareCaptionFont, captionsetup,
    % multirow
    multirow,
    % multicol
    columnbreak,
    % amssymb
    therefore, hdots,
    % mathtools
    prescript,
    % listings
    lstdefinestyle, lstset, lstinputlisting, lstlistlistingname, lstlistoflistings,
    % blindtext
    blindtext,
    % lipsum
    lipsum,
    % babel
    spanishtablename, captionsspanish,
    % titlesec
    titleformat, titleformat, titlespacing,
    % enumitem
    setlist,
    % hyperref
    hypersetup, url,
    % soulut8
    sethlcolor, hl,
    % Mis propios comandos.
    espacioleyenda, margenIzquierdo, margenDerecho,
    texttthl, instruccionlatex, hll
    }
}
\lstdefinestyle{latexi}{ % LaTeX inline
    style=latex,
    mathescape=false,
    basicstyle=\normalsize\ttfamily\color{codigo_base},
    columns=fixed,
    frame={},
}
% Texto rojo para resaltar errores.
\lstdefinestyle{errores}{
    basicstyle=\footnotesize\ttfamily\color{DarkRed!80!Black},
    commentstyle=\color{DarkRed!80!Black},
    keywordstyle=\color{DarkRed!80!Black},
    identifierstyle=\color{DarkRed!80!Black},
    mathescape=false, % Esto no puede ser true para este estilo, por el texto 'Missing $ inserted' (luego se marca un error)
}
% Texto amarillo para resaltar advertencias de compilación.
\lstdefinestyle{advertencias}{
    basicstyle=\footnotesize\ttfamily\color{DarkGoldenrod!70!Black},
    commentstyle=\color{DarkGoldenrod!70!Black},
    keywordstyle=\color{DarkGoldenrod!70!Black},
    identifierstyle=\color{DarkGoldenrod!70!Black},
    mathescape=false, % No marcar como true por uso de $.
}
% Definición del lenguaje BibTeX, para que tenga un poco de sintaxis.
\lstdefinelanguage{BibTeX}{
    keywords={%
        @tipo_de_fuente,@article,@book,@collectedbook,@conference,@electronic,@ieeetranbstctl,%
        @inbook,@incollectedbook,@incollection,@injournal,@inproceedings,%
        @manual,@mastersthesis,@masterthesis,@misc,@patent,@periodical,@phdthesis,@preamble,%
        @proceedings,@standard,@string,@techreport,@unpublished%
    },
    comment=[l]{\%},
    sensitive=false
}
% Definición del estilo BibTeX, para el capítulo de fuentes bibliográficas.
\lstdefinestyle{bibtex}{
    language=BibTeX,
    identifierstyle=\color{codigo_base},
    texcsstyle=*\bfseries\color{codigo_palabraclave},
    moretexcs={url}
}
% Permite usar las barras verticales para crear entornos en línea con el texto de estilo LaTeX.
%     Tomado de https://tex.stackexchange.com/questions/264293/cant-escape-curly-braces-in-lstinline
\lstMakeShortInline[style=latexi]|

%%%%%%%%%%%%%%%%%%%%%%%%%%%%%%%%%%%%%%%%%%%%%%%%%%%%%%%%%%%%%%%%%%%%%%%%%%%%%%%
%%%%% Configuración del paquete fancyhdr %%%%%%%%%%%%%%%%%%%%%%%%%%%%%%%%%%%%%%
%%%%%%%%%%%%%%%%%%%%%%%%%%%%%%%%%%%%%%%%%%%%%%%%%%%%%%%%%%%%%%%%%%%%%%%%%%%%%%%
\pagestyle{fancy}
\renewcommand{\chaptermark}[1]{\markboth{#1}{}}
\renewcommand{\headrule}{} % Remueve línea horizontal propia de fancyhdr.
\fancyhfoffset[L]{\margenAdicional}
\fancyhead[LE,CE,RE,LO,CO,RO]{} % Remover todo lo del encabezado.
\fancyfoot[LE,CE,RE,LO,CO,RO]{} % Remover todo lo del pie de página.
\fancyfoot[RO]{ % Para el pie de página de una página impar, a la derecha.
    \footnotesize % Todo el pie estará en este tamaño.
    \textcolor{PiePaginaCapitulo}{ % Color para nombre de capítulo.
        \leftmark % Imprimir el nombre del capítulo.
    } % Fin de cambio de color
    \kern10pt % Dejar un espacio de 10pt respecto a lo que siga.
    \textcolor{PiePaginaNumeroPagina}{ % Color para número de página.
        \textbf{\thepage} % Imprimir el número de página.
    } % Fin del otro cambio de color.
}
\fancyfoot[LE]{ % Para el pie de página de un página par, a la izquierda.
    \footnotesize % Todo el pie estará en este tamaño.
    \textcolor{PiePaginaNumeroPagina}{
        \textbf{\thepage} % Imprime el número de página.
    }
    \ifnum\value{chapter}>0 % \ifnum evita "Capítulo 0" en el índice.
        \kern10pt % Espacio entre número de página y otro texto.
        \textcolor{PiePaginaCapitulo}{
            \chaptername~\thechapter % Imprime "Capítulo" y número.
        }
    \fi % Fin del \ifnum.
}
\fancypagestyle{plain}{
    \fancyhead{} % Nada en el header.
    \renewcommand*{\headrule}{} % Eliminar la regla horizontal.
    \fancyfoot{} % Nada en el footer.
}

%%%%%%%%%%%%%%%%%%%%%%%%%%%%%%%%%%%%%%%%%%%%%%%%%%%%%%%%%%%%%%%%%%%%%%%%%%%%%%%
%%%%% Configuración del paquete fncychap %%%%%%%%%%%%%%%%%%%%%%%%%%%%%%%%%%%%%%
%%%%%%%%%%%%%%%%%%%%%%%%%%%%%%%%%%%%%%%%%%%%%%%%%%%%%%%%%%%%%%%%%%%%%%%%%%%%%%%

\renewcommand\DOCH{ % DOCH aplica para los capitulos con número y nombre.
\begin{adjustwidth}{-\margenIzquierdo-\margenAdicional}{-\margenDerecho}
    \nointerlineskip%
    \vspace{-1.625in}%
    \settowidth{\py}{\CNoV\thechapter}%
    \addtolength{\py}{\margenDerecho}%
    \addtolength{\py}{0.05in}%
    \fboxsep=0pt%
    \colorbox{FondoCapitulo}{\rule{0pt}{1.75in}\parbox[b]{1.01\paperwidth}{\hfill}}%
    \kern-\py\raise20pt%
    \hbox{\color{TextoCapitulo}\CNoV\thechapter}\\%
\end{adjustwidth}
}
\renewcommand\DOTI[1]{ % DOTI es para \chapter
\begin{adjustwidth}{-\margenIzquierdo-\margenAdicional}{0in}
    \nointerlineskip%
    \fboxsep=\myhi%
    \vskip-0.20in%
    \colorbox{FondoCapitulo}{\parbox[t]{\paperwidth}{\CTV\FmTi{\color{TextoCapitulo}#1}\kern\margenDerecho\kern0.055in}}\par\nobreak%
    \vskip-1ex%
    \colorbox{FondoCapitulo}{\rule{0pt}{10pt}\parbox[b]{\paperwidth}{\hfill}}%
    \vskip 30pt%
\end{adjustwidth}
}
\renewcommand\DOTIS[1]{ % DOTIS aplica para los capítulos sin número y sin nombre (\chapter*).
\begin{adjustwidth}{-\margenIzquierdo-\margenAdicional}{0in}
    \vspace{-1.85in}
    \fboxsep=0pt
    \colorbox{FondoCapitulo}{\rule{0pt}{120pt}\parbox[b]{1.01\paperwidth}{\hfill}}\\%
    \nointerlineskip%
    \fboxsep=\myhi%`
    \vskip-1ex%
    \colorbox{FondoCapitulo}{\parbox[t]{\paperwidth}{\CTV\FmTi{\color{TextoCapitulo}#1}\kern\margenDerecho}}\par\nobreak%
    \vskip-1ex%
    \colorbox{FondoCapitulo}{\rule{0pt}{10pt}\parbox[b]{\paperwidth}{\hfill}}%
    \vskip 30pt%
\end{adjustwidth}
}

%%%%%%%%%%%%%%%%%%%%%%%%%%%%%%%%%%%%%%%%%%%%%%%%%%%%%%%%%%%%%%%%%%%%%%%%%%%%%%%
%%%%% Ajustes de titlesec %%%%%%%%%%%%%%%%%%%%%%%%%%%%%%%%%%%%%%%%%%%%%%%%%%%%%
%%%%%%%%%%%%%%%%%%%%%%%%%%%%%%%%%%%%%%%%%%%%%%%%%%%%%%%%%%%%%%%%%%%%%%%%%%%%%%%

% Esta línea se activa en caso de no querer números en las secciones (usar en conjunto con secnumdepth)
% \titleformat{\section}{\normalfont\bfseries\Large}{}{0pt}{}{}
% Modifica el espacio entre el título y el texto que tiene debajo.
\titlespacing{\section}{-\margenAdicional}{*4}{*1.5}

% SE aplica una lógica similar para la subsección.
\titleformat*{\subsection}{\itshape\Large}
\titlespacing{\subsection}{0in}{*3}{*1.0}

%%%%%%%%%%%%%%%%%%%%%%%%%%%%%%%%%%%%%%%%%%%%%%%%%%%%%%%%%%%%%%%%%%%%%%%%%%%%%%%
%%%%% Configuración de otros paquetes %%%%%%%%%%%%%%%%%%%%%%%%%%%%%%%%%%%%%%%%%
%%%%%%%%%%%%%%%%%%%%%%%%%%%%%%%%%%%%%%%%%%%%%%%%%%%%%%%%%%%%%%%%%%%%%%%%%%%%%%%

% Configuración del enlaces (requiere hyperref).
\hypersetup{colorlinks, breaklinks, urlcolor=Enlaces, citecolor=Enlaces, linkcolor=Enlaces}

% Color del texto resaltado (requiere soulutf8).
% \sethlcolor{Resaltado} % Obtenido de https://tex.stackexchange.com/a/42964

% Configuración de las leyendas de figuras (requiere caption).
\DeclareCaptionFont{color_leyenda}{\color{Leyendas}}
\captionsetup{
    font={small,color_leyenda}, % Color y tamaño de la leyenda.
    skip=\espacioleyenda        % Espacio entre leyenda y figura. Default = 10pt.
}

% Ajuste del interlineado (requiere setspace).
\setstretch{1.1875}

% Elimina el espacio entre elementos de una lista (requiere enumitem).
\setlist{noitemsep}

%%%%%%%%%%%%%%%%%%%%%%%%%%%%%%%%%%%%%%%%%%%%%%%%%%%%%%%%%%%%%%%%%%%%%%%%%%%%%%%
%%%%% Otras configuraciones que afectan el comportamiento de LaTeX %%%%%%%%%%%%
%%%%%%%%%%%%%%%%%%%%%%%%%%%%%%%%%%%%%%%%%%%%%%%%%%%%%%%%%%%%%%%%%%%%%%%%%%%%%%%

% Longitud del sangrado de párrafos.
\setlength\parindent{0.25in}

% No numerar secciones (los mantenemos en el libro para dar apariencia de tesis, donde todo va numerado).
% \setcounter{secnumdepth}{0}
% Numerar capítulos y secciones solamente (pero no subsecciones y debajo).
\setcounter{secnumdepth}{1}

% Nombre de tabla de contenido.
\addto\captionsspanish{\renewcommand{\contentsname}{Contenido}}
% \setcounter{tocdepth}{1} % Capítulos y secciones en el índice.
\setcounter{tocdepth}{2} % Capítulos, secciones, y subsecciones en el índice.

% No "justificar" verticalmente (que LaTeX no busque que la última línea de todas las páginas quede donde mismo).
\raggedbottom

% Valores que afectan a cómo LaTeX maneja las líneas huérfanas (líneas que alcanzan a cruzar a otras páginas, solas).
\widowpenalty=10000
\clubpenalty=10000

%%%%%%%%%%%%%%%%%%%%%%%%%%%%%%%%%%%%%%%%%%%%%%%%%%%%%%%%%%%%%%%%%%%%%%%%%%%%%%%
%%%%% Nuevos comandos %%%%%%%%%%%%%%%%%%%%%%%%%%%%%%%%%%%%%%%%%%%%%%%%%%%%%%%%%
%%%%%%%%%%%%%%%%%%%%%%%%%%%%%%%%%%%%%%%%%%%%%%%%%%%%%%%%%%%%%%%%%%%%%%%%%%%%%%%

% Comando para encapsular texto que hace referencia a una opción de un menú de interfaz.
\newcommand{\opcionMenu}[1]{\emph{\texttt{#1}}}
% Puntos verticales a ser usados en los listados.
\newcommand{\puntitoscodigo}{\scalebox{0.80}{\vdots}}
% Parche para la doble diagonal en línea con el texto.
\newcommand{\diagonalesCodigo}{\textcolor{codigo_palabraclave}{\textbf{\texttt{\textbackslash\textbackslash}}}}
\newcommand{\codigo}[1]{\textcolor{codigo_palabraclave}{\textbf{\texttt{\textbackslash{}#1}}}}
\newcommand{\comentario}[1]{\textcolor{codigo_comentarios}{{\hypersetup{linkcolor=codigo_comentarios}\emph{\texttt{#1}}}}}
% Parte de la explicación de instrucciones.
\newcommand{\texttthl}[1]{\texttt{\hl{#1}}}
\newcommand{\ttlatex}[1]{\texttt{\textbackslash{}#1}}
\newcommand{\instruccionlatex}[1]{\texttthl{\textbackslash{}#1}}

\newcommand{\BibTeX}{\textsc{Bib}\TeX{}}
\newcommand{\BibLaTeX}{\textsc{Bib}\LaTeX{}}

\newcommand{\fuenteOverleaf}[1]{\footnote{Ejemplo en Overleaf, en \href{http://overleaf.ramoscarlos.com}{http://overleaf.ramoscarlos.com}, archivo \texttt{#1}.}}

% Comando para hacer capítulo sin numerar, pero que aparezca en el índice.
\newcommand\chap[1]{
    \cleardoublepage
    \phantomsection
    \addcontentsline{toc}{chapter}{#1}
    \chapter*{#1}
}

\newcommand{\lstrulet}{\noindent\textcolor{codigo_linea_margen}{\rule{\textwidth}{0.4pt}}\newline}
\newcommand{\lstruleb}{\noindent\textcolor{codigo_linea_margen}{\rule{\textwidth}{0.4pt}}}