\chap{Comentarios finales}



A decir verdad, tuve que ser detenido para no escribir más. Varias de las personas que me ayudaron con la edición me comentaron que ya era suficiente contenido, aunque a mí no me lo parece.

¡Apenas estábamos comenzando!

Falta la explicación de la distribución del documento, cómo calcula \LaTeX{} sus márgenes, y una larga discusión teniendo como base los esqueletos que se pueden generar con el paquete \texttt{layouts}.

No obstante, cuando se me encaró directamente sobre si eso era necesario para que un estudiante de ingeniería redactara su tesis, tuve que decir la verdad: no.

Siento que mucho contenido fue dejado fuera injustamente, que \LaTeX{} es un gran tema para una conversación en un café, acompañado de la brisa matinal... entre quien disfruta del tema, claro.

De otra forma, ha llegado la hora de partir. El conocimiento suficiente para redactar una tesis se ha transmitido, y todo lo demás es vanidad.

Además, lo bueno de que este documento sea principalmente destinado a formato digital quiere decir que se puede modificar iterativamente, agregando contenido según lo pidan las hordas de estudiantes que consulten este material (vamos, pregunta, pregunta).

Sin más por el momento, espero que la presente obra te haya dado luz en el camino de la redacción de tu tesis. También espero, sinceramente, que concluyas tus estudios de una manera satisfactoria.

¡Hasta pronto!