\chap{Agradecimientos}
\label{cha:agradecimientos}



Había querido escribir este libro desde hace más de cinco años, idea que surgió cuando tuve que volver a usar \LaTeX{} para mi tesis de maestría y me encontraba frustrado con mis problemas de formato. Me dije a mí mismo que haría una guía para aquellos que entraban a este sistema como yo, perdidos y obligados, pero nunca había reunido el valor suficiente para plasmar mis ideas en un documento digital.

A lo largo de todos estos años la idea no me abandonó, lo que finalmente dió como resultado este libro. Ha pasado mucho tiempo, pero aún no olvido el motivo por el que conocí \LaTeX{}: fue gracias a mi asesor de tesis a nivel licenciatura, el Doctor Abimael Jiménez, quien me sugirió utilizar este programa desconocido para mí, sin una mayor guía que una bendición académica.

Al principio, la sugerencia carecía de sentido. Lo único que encontraba eran errores. Suficiente tenía con tratar de terminar mi tesis como para también tener conflictos al tratar de documentar mis vanos intentos de graduarme.

No obstante, la paciencia y comprensión de mi asesor fue pieza clave en la consecución de los objetivos y quedé tan satisfecho con el resultado que volví al mundo académico por una maestría.

Ya allí comprendí que la ingeniería no había sido nada, y que el conocimiento de \LaTeX{} me facilitaría mucho mi existencia (para escribir sobre matrices y cálculo variacional). En esta etapa hubo otro doctor que ocasionó traumas severos que fomentaron mi desarrollo académico, un doctor llamado Antonio Muñoz. Gracias por la presión.

Suficiente del ámbito académico. Me llevó más de dos décadas comprender que no sería nada sin el apoyo de mi familia, tanto en el plano económico como en el plano emocional. A pesar de las desavenencias, he de admitir que la guía de mis padres, Alberto y Lorenza, me ha conducido hasta esta obra, a este gusto por aprender más, por leer más, y por querer transmitir este conocimiento para que las nuevas generaciones desarrollen frustraciones diferentes.

No puedo olvidar a mi hermana, Miriam, que tuvo que cargar con la rebeldía de un ser tres años menor que ella cuyo objetivo parecía ser meterse en todas las actividades que le prohibieron. Posiblemente por eso estoy aquí ahora como un escritor técnico, en lugar de estar en alguna fosa común, un destino alarmantemente habitual en el barrio donde fui criado.

También he de agradecer a Daniel Morales, por su paciencia como editor y amigo. Estoy seguro que ya ha leído este libro cinco veces mínimo, con todas las revisiones y correcciones que me ha sugerido, iteración tras iteración.

Finalmente, me gustaría agradecer a Ixchel Franco, cuya paciencia con esta primera obra ha sido magistral. En mi estado escritor me olvido de mí mismo por horas hasta que, por algún milagro, me despego de la computadora por un segundo y me doy cuenta que he pasado ocho horas sin comer. Por fortuna, mi prometida es una persona sensata y ya sabe que si no le respondo es porque entré en algún trance de inspiración o investigación, en un problema que me obsesiona y debe ser resulto a la de ayer.

Y hay muchas más personas involucradas en este proyecto, pero no quiero extenderme mucho más. Ahora ya sé cómo es que los autores escriben y escriben en esta sección. Antes me preguntaba si era para rellenar el espacio, aunque ahora me doy cuenta de la enorme tarea que un libro supone, y de todo el soporte que se necesita para terminar un proyecto así.

Algunos días, incluso el simple comentario de que siguiera escribiendo por parte de mis amistades cercanas era suficiente para dispersar los pensamientos de derrotismo, aquellos episodios donde me preguntaba para qué seguir si nadie leería mi obra.

A todos aquellos que me dieron su aliento, que comprendieron que este proyecto demandaba de mi tiempo y me dieron espacio... gracias.

Ahora, antes de mi treinta y un aniversario... por fin he cumplido mi sueño de convertirme en un escritor (con una obra registrada).


