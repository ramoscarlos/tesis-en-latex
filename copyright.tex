% Copyright y más textos de derechos.
{
\newpage % Empezar en una nueva página.
\thispagestyle{empty} % Remover numeración de esta página solamente.

\begin{adjustwidth}{-\margenAdicional}{}

\vspace*{1cm}

\fontsize{16pt}{0pt}\selectfont
\noindent\textbf{\LaTeX{} para tesis de ingeniería}\\
\fontsize{14pt}{0pt}\selectfont\vspace{4pt}
\newline\emph{Carlos Alberto Ramos López}\\

\vspace*{2cm}
\fontsize{10pt}{0pt}\selectfont

\noindent Este libro se encuentra a la venta en\\
\vspace{2pt}\newline\url{https://leanpub.com/tesis-en-latex/}

\vspace*{2cm}

\noindent Esta versión se publicó el \today.

\vspace*{1cm}

\noindent\begin{minipage}[t]{2cm}
\includegraphics[width=\linewidth]{img/LeanPub.png}
\end{minipage}
\begin{minipage}[t]{\linewidth-2cm}
\vspace{-41pt}
Este es un libro publicado en la plataforma de \href{https://leanpub.com/}{LeanPub}. Aunque este libro no fue generado mediante su sistema, incluyo esta información porque me parece una plataforma justa, que merece el crédito por apoyar a los autores con obras en progreso (o proclives a muchas actualizaciones, como lo son libros relacionados con tecnologías de la información).
\end{minipage}

\vspace*{1cm}

\noindent\begin{minipage}[t]{2cm}
\includegraphics[width=\linewidth]{img/CC-BY-NC-SA.png}
\end{minipage}
\begin{minipage}[t]{\linewidth-2cm}
\vspace{-19.5pt}
Este trabajo está licenciado bajo \href{https://creativecommons.org/licenses/by-nc-sa/4.0/deed.es}{Creative Commons BY-NC-SA 4.0}, lo que implica que eres libre de compartir y redistribuir esta obra, siempre y cuando:
\begin{itemize}[nosep]
    \item Des el crédito de manera adecuada a la obra original con un enlace a la página \url{http://latex.ramoscarlos.com}.
    \item No hagas uso comercial de la obra (puedes compartirla gratis, pero no venderla).
    \item Si compartes la obra, lo hagas bajo esta misma licencia.
\end{itemize}
\end{minipage}

\end{adjustwidth}
}